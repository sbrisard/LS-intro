\chapter{On the d-dimensional brick element}

In this chapter, we formulate the linear brick finite element for conductivity
and linear elasticity in \(d\) dimensions (\(d\in\{2, 3\}\)).

\begin{remark}
  In the remainder of this chapter, \(i\), \(j\), \(h\) and \(k\) denote scalar
  indices that refer to tensor components, and span \(\{1, 2, \cdots d\}\);
  \(\tuple{ p}\), \(\tuple{q}\) are \(d\)-dimensional multi-indices that refer
  to nodes. Depending on the context (local, element operators or global,
  assembled operators) they span \(\{1, 2\}^d\) or the whole grid
  \(\cellindices\). We adopt Einstein's convention for indices of both types.
\end{remark}

\section{Geometry of the reference brick element}

The reference element being centered at the origin, it occupies the following
domain
\begin{equation}
  \Omega^\element=\biggl(-\frac{h_1}2, \frac{h_1}2\biggr)\times\cdots
  \times\biggl(-\frac{h_d}2, \frac{h_d}2\biggr),
\end{equation}
where \(h_i\) is the size of the element along axis \(i\) (\(i=1, \dots,
d\)). The nodes of the element are indexed by \(\tuple{p}\in\{1, 2\}^d\), such
that the coordinates of node \(\tuple{p}\) read
\begin{equation}
  x_{i\tuple{p}}=(-1)^{p_i}\frac{h_i}2.
\end{equation}

For integration purposes, it is useful to introduce the \emph{reduced
  coordinates} \(\xi_i=2x_i/h_i\)
\begin{equation}
  \int_{\Omega^\element}f(x_1, \ldots, x_d)\,\D x_1\cdots\D x_d
  =\frac{h_1\cdots h_d}{2^d}\int_{(-1, 1)^d}f(h_1\xi_1/2, \ldots, h_d\xi_d/2)
  \,\D\xi_1\cdots\D\xi_d
\end{equation}
and it is observed that the reduced coordinates of the node \(\tuple{p}\) are
\(\xi_i=(-1)^{p_i}\).

\section{Shape functions}

The shape function associated with node \(\tuple{p}\) is
\(N^\element_{\tuple{p}}(\vec x)\), wich is defined below as a function of the
reduced coordinates of the observation point
\begin{equation}
  N^\element_{\tuple{p}}(\vec x)=\frac1{2^d}\prod_{i=1}^d\bigl[1+(-1)^{p_i}\xi_i\bigr]
\end{equation}
and we have as expected
\(N^\element_{\tuple{p}}(\vec x_{\tuple q}) = \delta_{\tuple{p}\tuple{q}}\). The
nodal values of \(f\) being \(f_\tuple{p}\), we have the interpolation formula:
\(f(\vec x)=N^\element_\tuple{p}(\vec x)f_\tuple{p}\).

\section{Gradient and strain-displacement operators}

We consider a scalar interpolated field,
\(f(\vec x)=N^\element_\tuple{p}(\vec x)f_\tuple{p}\). The components of its
gradient are given by the following expression
\begin{equation}
  \bigl(f\otimes\nabla\bigr)_i=\partial_i f=D_{i\tuple{p}}^\element f_\tuple{p},
\end{equation}
where the observation point \(\vec x\) has been dropped, and
\(D_{i\tuple{p}}^\element\) denotes the element gradient operator
\begin{equation}
  D_{i\tuple{p}}^\element=\partial_iN_\tuple{p}
  =\frac{(-1)^{p_i}}{2^{d-1}h_i}
  \prod_{j\neq i}\bigl[1+(-1)^{p_j}\xi_j\bigr].
\end{equation}

It will be useful to introduce the element average of this operator,
\(\overline{D}_{i\tuple{n}}^\element\)
\begin{equation}
  \overline{D}_{i\tuple{p}}^\element=\frac 1{h_1\cdots h_d}
  \int_{\Omega^\element}D^\element_{i\tuple{p}}(\vec x)\,\D x_1\cdots\D x_d
\end{equation}
and, observing that each factor in square brackets in the above expression of
\(D^\element_{i\tuple{p}}\) varies linearly between 0 et 2 (and therefore
averages to 1), we find
\begin{equation}
  \overline{D}^\element_{i\tuple{p}}=\frac{(-1)^{p_i}}{2^{d-1}h_i}.
\end{equation}

We now consider the interpolated displacement field,
\(\vec u = N^\element_\tuple{p} \vec u_\tuple{p}\). The components of the
associated strain, \(\tens\varepsilon=\vec u\symotimes\nabla\) are
\begin{equation}
  \varepsilon_{ij}=\tfrac12\bigl(\partial_iu_j+\partial_ju_i\bigr)
  =\tfrac12\bigl(D^\element_{i\tuple{p}}u_{j\tuple{p}}
  +D^\element_{j\tuple{p}}u_{i\tuple{p}}\bigr),
\end{equation}
where \(u_{i\tuple{p}}\) denotes the \(i\)-th component of the nodal value of
\(\vec u\) at node \(\tuple{p}\),
\(u_{i\tuple{p}}=\vec e_i\cdot\vec u(\vec x_\tuple{p})\).

\begin{remark}
  In the Julia implementation of the brick element, the nodal displacements are
  stored in a \(d+1\)-dimensional array \code{u}, such that
  \(\text{\code{u[i, p]}}=u_{i\tuple{p}}\), where \code{i~in~\{1, ... d\}} and
  \code{p~in~CartesianIndices(1:N[1], …, 1:N[d])}. With this convention, the
  component index, \code{i}, is the \emph{fast} index. Therefore, all dofs of
  the same node are contiguous, which is the usual convention in finite element.
\end{remark}

Rearranging the above expression, we find
\(\varepsilon_{ij}=B^\element_{ijk\tuple{p}}u_{k\tuple{p}}^\element\), where
\begin{equation}
  B_{ijk\tuple{p}}^\element
  =\tfrac12\bigl(D^\element_{i\tuple{p}}\delta_{jk}
  +D^\element_{j\tuple{p}}\delta_{ik}\bigr)
\end{equation}
is the element strain-displacement operator, with volume average
\begin{equation}
  \overline{B}_{ijk\tuple{p}}^\element
  =\tfrac12\bigl(\overline{D}^\element_{i\tuple{p}}\delta_{jk}
  +\overline{D}^\element_{j\tuple{p}}\delta_{ik}\bigr)
  =\frac1{2^d}\Bigl[\frac{(-1)^{p_i}\delta_{jk}}{h_i}
  +\frac{(-1)^{p_j}\delta_{ik}}{h_j}\Bigr].
\end{equation}

It is obverved that the \emph{trace} of the strain tensor is given by the
following expression
\begin{equation}
  \varepsilon_{ii}=D_{i\tuple{p}}u_{i\tuple{p}}^\element.
\end{equation}

\section{Element stiffness operator}

\subsection{Conductivity}

\subsection{Linear elasticity}

Assuming the material to be linearly elastic, with Lamé coefficients \(\lambda\)
and \(\mu\), we have the following expression of the interpolated stresses as a
function of the nodal displacements \(u_{i\tuple{p}}\)
\begin{equation}
  \sigma_{hk}=\bigl(\lambda\delta_{hk} D^\element_{i\tuple{p}}
  +2\mu B^\element_{hki\tuple{p}}\bigr)u_{i\tuple{p}}
\end{equation}
and the volume density of elastic energy reads
\begin{equation}
  \tfrac12\sigma_{hk}\varepsilon_{hk}
  =\bigl(\lambda\delta_{hk} D^\element_{i\tuple{p}}
  +2\mu B^\element_{hki\tuple{p}}\bigr)B_{hkj\tuple{q}}^\element
  u_{i\tuple{p}}u_{j\tuple{q}}.
\end{equation}

Upon expansion and integration, we get the expression of the elastic energy
within the element
\begin{equation}
  U^\element
  =\int_{\Omega^\element}\tfrac12\sigma_{hk}\varepsilon_{hk}\,\D x_1\cdots\D x_d
  =\tfrac12K^\element_{i\tuple{p}j\tuple{q}}u_{i\tuple{p}}u_{j\tuple{q}},
\end{equation}
where the element stiffness operator reads
\begin{equation}
  K^\element_{i\tuple{p}j\tuple{q}}
  =\lambda\,K^{\element, \lambda}_{i\tuple{p}j\tuple{q}}
  +\mu\,K^{\element, \mu}_{i\tuple{p}j\tuple{q}},
\end{equation}
with
\begin{equation}
  K^{\element, \lambda}_{i\tuple{p}j\tuple{q}}
  =\int_{\Omega^\element}D^\element_{i\tuple{p}}D^\element_{j\tuple{q}}
  \quad\text{and}\quad
  K^{\element, \mu}_{i\tuple{p}j\tuple{q}}
  =\int_{\Omega^\element} 2 B^\element_{hki\tuple{p}}B^\element_{hkj\tuple{q}}.
\end{equation}
