\chapter{Nomenclature}
\label{cha:20210924084856}

\begin{itemize}
\item \(d\): dimension of the physical space (typically \(d=2, 3\))
\item \(\Omega\): \(d\)-dimensional unit-cell
\item \(L_1,\ldots, L_d\): dimensions of the unit-cell:
  \(\Omega=(0, L_1)\times(0, L_2)\times\cdots\times(0, L_d)\)
\item \(\lvert\Omega\rvert=L_1L_2\cdots L_d\): volume of the unit-cell
\item \(\tuple{n}\): \(d\)-dimensional tuple of integers
  \(\tuple{n}=(n_1, n_2, \ldots, n_d)\)
\item \(\tuple{N}=(N_1, N_2, \ldots, N_d)\): size of the simulation grid
\item \(\lvert N\rvert=N_1N_2\cdots N_d\): total number of cells
\item \(h_i=L_i/N_i\): size of the cells (\(i=1, \ldots, d\))
\item \(\cellindices=\{1, \ldots, N_1\}\times\cdots\times\{1, \ldots, N_d-1\}\):
  set of cell indices
\item \(\Omega_{\tuple{p}}\): cells of the simulation grid
  (\(\tuple{p}\in\cellindices\))
\end{itemize}

\begin{remark}[Naming conventions for indices]
  \label{rem:20210924090334}
  In this book, we try to stick to the following conventions
  \begin{itemize}
  \item italic, latin indices (\(i\), \(j\), \ldots) refer to components of a
    tensor in the \(d\)-dimensional space,
  \item upright, sans-serif, indices are multi-indices over a cartesian grid,
  \item \(\tuple{p}\), \(\tuple{q}\) are multi-indices in the real space
    (``pixels''),
  \item \(\tuple{m}\), \(\tuple{n}\) are multi-indices in the frequency space.
  \end{itemize}
\end{remark}

\begin{remark}[Indexing conventions]
  \label{rem:20210924090344}
  In the above definition of \(\cellindices\), it is observed that multi-indices
  start at 1. This is in line with the Julia\footnote{It is recalled that
    Lippmann--Schwinger solvers presented in the present book are implemented in
    the library \href{https://github.com/sbrisard/Scapin.jl}{Scapin.jl}}
  programming language, which is 1-based.
\end{remark}
