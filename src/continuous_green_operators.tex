\chapter{Continuous Green operators}

In this chapter, we discuss various boundary-value problems in a periodic
setting. For each of these problems, we introduce the associated *Green
operator*.

\section{Conductivity}

\section{Continuous Green operator for linear elasticity}

We first define a few functional spaces; \(\tensors_2(\Omega)\) denotes the
space of second-order, symmetric, tensor fields, with square-integrable
components. Then, the space \(\stresses(\Omega)\) of periodic, self-equilibrated
stresses is defined as follows
\begin{equation}
  \tens\sigma\in\stresses(\Omega)\iff\left\{
    \begin{gathered}
      \tens\sigma\in\tensors_2(\Omega)\\
      \tens\sigma\cdot\nabla=\vec 0\text{ a.e in }\Omega\\
      \tens\sigma\cdot\vec e_i\text{ is }L_i\vec e_i
      \text{-periodic for all }i=1, 2, \ldots, d\text{ (no summation),}
    \end{gathered}
  \right.
\end{equation}
where the last condition expresses the periodicity of tractions in all
directions parallel to the sides of the unit-cell. The space
\(\tens\strains(\Omega)\) of periodic, geometrically compatible strains is
defined as follows
\begin{equation}
  \tens\varepsilon\in\strains(\Omega)\iff\left\{
    \begin{gathered}
      \tens\varepsilon\in\tensors_2(\Omega)\\
      \tens\varepsilon=\vec u\symotimes\nabla
      \text{ a.e. in }\Omega\text{ for some vector field }\vec u\\
      \vec u\text{ has square-integrable components}\\
      \vec u\text{ is }\Omega\text{-periodic.}
    \end{gathered}
  \right.
\end{equation}

Finally, we define the spaces of stresses and strains with zero average
\begin{equation}
  \stresses_0(\Omega)=\bigl\{\tens\sigma\in\stresses(\Omega),
  \langle\tens\sigma\rangle=\tens0\bigr\}
  \quad\text{and}\quad
  \strains_0(\Omega)=\{\tens\varepsilon\in\strains(\Omega),
  \langle\tens\varepsilon\rangle=\tens0\bigr\}.
\end{equation}

We are now ready to define the periodic, fourth-order Green operator for strains
\(\tens\Gamma\). Let \(\tens C\) be the homogeneous elastic stiffness of the
body \(\Omega\)\footnote{In other words, \(\tens C\) is a constant, fourth-order
  tensor with major and minor symmetries; furthermore, \(\tens C\) is positive
  definite.}. Let \(\tens\tau\in\tensors_2(\Omega)\) be a prescribed tensor
field (\emph{stress-polarization}). We want to find the equilibrium state of the
body \(\Omega\), subjected to the eigenstress \(\tens\tau\) and periodic boundary
conditions. In other words, we want to find the solution to the following
problem
\begin{equation}
  \text{Find }\tens\sigma\in\stresses_0(\Omega)
  \text{ and }\tens\varepsilon\in\strains_0(\Omega)
  \text{ such that }\tens\sigma=\tens C\dbldot\tens\varepsilon+\tens\tau
  \text{ a.e. in }\Omega.
\end{equation}

Owing to the periodic boundary conditions, we use Fourier series expansions of
\(\tens\tau\), \(\tens\sigma\), \(\tens\varepsilon\) and \(\vec u\) (see
Sec.~\ref{sec:20210924084626})
\begin{equation}
  \begin{Bmatrix}
    \tens\tau(\vec x)\\
    \tens\sigma(\vec x)\\
    \tens\varepsilon(\vec x)\\
    \vec u(\vec x)
  \end{Bmatrix}
  =\sum_{\tuple{n}\in\integers^d}
  \begin{Bmatrix}
    \tilde{\tens\tau}_\tuple{n}\\
    \tilde{\tens\sigma}_\tuple{n}\\
    \tilde{\tens\varepsilon}_\tuple{n}\\
    \tilde{\vec u}_\tuple{n}
  \end{Bmatrix}
  \exp(\I \vec k_\tuple{n}\cdot\vec x).
\end{equation}

The Fourier modes \(\tilde{\tens\sigma}_\tuple{n}\),
\(\tilde{\tens\varepsilon}_\tuple{n}\) and \(\tilde{\vec u}_\tuple{n}\) solve
the following equations (respectively: equilibrium, geometric compatibility,
constitutive relation)
\begin{gather}
  \label{eq:20210927092532}
  \tilde{\tens\sigma}_\tuple{n}\cdot\vec k_\tuple{n}=\vec 0\\
  \label{eq:20210927092538}
  \tilde{\tens\varepsilon}_\tuple{n}
  =\I\,\tilde{\vec u}_\tuple{n}\symotimes\vec k_\tuple{n}\\
  \label{eq:20210927092543}
  \tilde{\tens\sigma}_\tuple{n}=\tens C\dbldot\tilde{\tens\varepsilon}_\tuple{n}
  +\tilde{\tens\tau}_\tuple{n}.
\end{gather}

Plugging the Eq.~\eqref{eq:20210927092543} into Eq.~\eqref{eq:20210927092538},
and recalling that \(\tens C\) has the minor symmetries, we find the following
expression of \(\tilde{\tens\sigma}\)
\begin{equation}
  \tilde{\tens\sigma}_\tuple{n}
  =\I\bigl(\tens C\cdot\vec k_\tuple{n}\bigr)\cdot\tilde{\vec u}_\tuple{n}
  +\tilde{\tens\tau}_\tuple{n}.
\end{equation}

The Cauchy stress tensor being symmetric, Eq.~\eqref{eq:20210927092532} also
reads \(\vec k_\tuple{n}\cdot\tilde{\tens{\sigma}}_\tuple{n}=\vec 0\) and
\begin{equation}
  \tilde{\vec u}_\tuple{n}
  =\I\bigl(\vec k_\tuple{n}\cdot\tens C\cdot\vec k_\tuple{n}\bigr)^{-1}
  \cdot\tilde{\tens\tau}_\tuple{n}\cdot\vec k_\tuple{n},
\end{equation}
which delivers the following expression for the Fourier modes of the strain
field
\begin{equation}
  \tilde{\tens\varepsilon}_\tuple{n}
  =-\bigl[\bigl(\vec k_\tuple{n}\cdot\tens C\cdot\vec k_\tuple{n}\bigr)^{-1}
  \cdot\tilde{\tens\tau}_\tuple{n}\cdot\vec k_\tuple{n}\bigr]
  \symotimes\vec k_\tuple{n}.
\end{equation}
The above relation defines a linear mapping between
\(\tilde{\tens\tau}_\tuple{n}\) and \(\tilde{\tens\varepsilon}_\tuple{n}\). For
each Fourier mode \(\tuple{n}\), we therefore introduce the fourth-order tensor
\(\tilde{\tens\Gamma}_\tuple{n}\) with major and minor symmetries, such that
\(\tilde{\tens\varepsilon}_\tuple{n}=-\tilde{\tens\Gamma}_\tuple{n}\dbldot{\tilde{\tens\tau}}_\tuple{n}\). Our
analysis shows that
\(\tilde{\tens\Gamma}_\tuple{n}=\hat{\tens\Gamma}(\vec k_\tuple{n})\) where, for
arbitrary wave-vector \(\vec k\), \(\hat{\tens\Gamma}(\vec k)\) is a
fourth-order tensor with major and minor symmetries, such that
\begin{equation}
  \hat{\tens\Gamma}(\vec k)\dbldot\tilde{\tens\tau}
  =\bigl[\bigl(\vec n\cdot\tens C\cdot\vec n\bigr)^{-1}
  \cdot\tilde{\tens\tau}\cdot\vec n\bigr]\symotimes\vec n,
\end{equation}
where \(\vec n=\vec k/\lVert\vec k\rVert\). The above equation defines
\(\hat{\tens\Gamma}(\vec k)\) by how it operates on second-order, symmetric
tensors. A closed-form expression of this tensor can be derived in the case of
an isotropic material, for which
\begin{equation}
  \tens C=\lambda\tens I_2\otimes\tens I_2+2\mu\tens I_4,
\end{equation}
where \(\tens I_2\) (resp. \(\tens I_4\)) is the second-order
(resp. fourth-order) identity tensor, and \(\lambda\), \(\mu\) are the Lamé
coefficients. Then
\begin{equation}
  \vec n\cdot\bigl(\tens I_2\otimes\tens I_2\bigr)\cdot\vec n=\vec n\otimes\vec n
\end{equation}
and (recalling that \(\lVert\vec n\rVert=1\))
\begin{equation}
  \begin{aligned}[b]
    \vec n\cdot\tens I_4\cdot\vec n
    &=\tfrac12 n_i\bigl(\delta_{ik}\delta_{jl}+\delta_{il}\delta_{jk}\bigr)n_l
    \vec e_j\otimes\vec e_k
    =\tfrac12\bigl(n_kn_j+n_in_i\delta_{jk}\bigr)\vec e_j\otimes\vec e_k\\
    &=\tfrac12\bigl[\vec n\otimes\vec n+\bigl(\vec n\cdot\vec n\bigr)\tens I_2\bigr]
    =\tfrac12\bigl(\vec n\otimes\vec n+\tens I_2\bigr)
    =\vec n\otimes\vec n+\tfrac12\bigl(\tens I_2-\vec n\otimes\vec n\bigr)
  \end{aligned}
\end{equation}
and finally, we find the following expression of the acoustic tensor
\begin{equation}
  \vec n\cdot\tens C\cdot\vec n
  =\bigl(\lambda+2\mu\bigr)\vec n\otimes\vec n
  +\mu\bigl(\tens I_2-\vec n\otimes\vec n\bigr)
  =2\mu\frac{1-\nu}{1-2\nu}\vec n\otimes\vec n
  +\mu\bigl(\tens I_2-\vec n\otimes\vec n\bigr),
\end{equation}
where \(\nu\) denotes the Poisson ratio. The above second-order tensor is easily
inverted, since \(\vec n\otimes\vec n\) and \(\tens I_2-\vec n\otimes\vec n\)
are two orthogonal projectors (in the sense of the ``\(\dbldot\)'' product)
\begin{equation}
  2\mu\bigl(\vec n\cdot\tens C\cdot\vec n\bigr)^{-1}
  =\frac{1-2\nu}{1-\nu}\vec n\otimes\vec n
  +2\bigl(\tens I_2-\vec n\otimes\vec n\bigr)
  =2\tens I_2-\frac{\vec n\otimes\vec n}{1-\nu},
\end{equation}
from which it results that
\begin{equation}
  2\mu\bigl(\vec n\cdot\tens C\cdot\vec n\bigr)^{-1}
  \cdot\tilde{\tens\tau}\cdot\vec n
  =2\tilde{\tens\tau}\cdot\vec n
  -\frac{\vec n\cdot\tilde{\tens\tau}\cdot\vec n}{1-\nu}\vec n
\end{equation}
and we finally get
\begin{equation}
  \label{eq:20210927093916}
  2\mu\hat{\tens \Gamma}(\vec k)\dbldot\tilde{\tens \tau}
  =\bigl(\tilde{\tens \tau}\cdot\vec n\bigr)\otimes\vec n
  +\vec n\otimes\bigl(\tilde{\tens \tau}\cdot\vec n\bigr)
  -\frac{\vec n\cdot\tilde{\tens \tau}\cdot\vec n}{1-\nu}\vec n\otimes\vec n.
\end{equation}

The components of the \(\hat{\tens \Gamma}\) tensor are then readily found
\begin{equation}
  \hat{\Gamma}_{ijkl}(\vec k)
  =\frac{\delta_{ik}n_jn_l+\delta_{il}n_jn_k+\delta_{jk}n_in_l+\delta_{jl}n_in_k}{4\mu}
  -\frac{n_in_jn_kn_l}{2\mu\bigl(1-\nu\bigr)},
\end{equation}
which coincide with classical expressions (see
e.g. \textcite{suqu1990}). Implementation of the above equation is cumbersome;
it is only used for testing purposes. In Scapin, only the \emph{matvec} product
is required, and Eq.~\eqref{eq:20210927093916} was implemented.

\section{Hyperelasticity}
