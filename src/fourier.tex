\chapter{Fourier transforms in a periodic setting}

This chapter provides a brief overview of the two Fourier transforms that are
going to be used in the previous document, namely Fourier series (see
Sec.~\ref{sec:20210924084626}) and discrete Fourier transforms (see
Sec.~\ref{sec:20210924084718}).

\section{Fourier series}
\label{sec:20210924084626}

Owing to the periodic setting, the fields that are involved in the various BVPs
to be discussed in this document are expanded in Fourier series. \(\tens T\)
being a \(\Omega\)-periodic tensor field (with sufficient regularity), the
following decomposition holds
\begin{equation}
  \tens T(\vec x)
  =\sum_{\tuple{n}\in\integers^d}\mathcal F(\tens T)(\vec k_{\tuple{n}})
  \exp(\I\vec k_{\tuple{n}}\cdot\vec x),
\end{equation}
where \(\tuple{n}\) denotes a \(d\)-dimensional tuple of integers (see
nomenclature, Chap.~\ref{cha:20210924084856}). The wave vectors
\(\vec k_{\tuple{n}}\) are given by
\begin{equation}
  \vec k_{\tuple{n}}=\frac{2\pi n_1}{L_1}\vec e_1
  +\frac{2\pi n_2}{L_2}\vec e_2
  +\cdots+\frac{2\pi n_d}{L_d}\vec e_d,
\end{equation}
and the Fourier coefficients of \(\tens T\) are defined as follows
\begin{equation}
\mathcal F(\tens T)(\vec k)=\frac1{\lvert\Omega\rvert}\int_{\vec x\in\Omega}
\tens T(\vec x)\exp(-\I\vec k\cdot\vec x)\,\D x_1\cdots\D x_d.
\end{equation}

It is recalled that the Fourier coefficients of the gradient and divergence of
\(\tens T\) can readily be computed from the Fourier coefficients of \(\tens T\)
\begin{equation}
\mathcal F(\tens T\otimes\nabla)(\vec k)=\mathcal F(\tens T)(\vec k)\otimes\I\vec k
\quad\text{and}\quad
\mathcal F(\tens T\cdot\nabla)(\vec k)=\mathcal F(\tens T)(\vec k)\cdot\I\vec k.
\end{equation}

When no confusion is possible, we will use a tilde to denote the Fourier
coefficients:
\(\tilde{\tens T}_\tuple{n}=\mathcal F(\tens T)(\vec k_\tuple{n})\).

\section{Discrete Fourier transforms}
\label{sec:20210924084718}

We consider a \(d\)-dimensional grid of size \(N_1\times\cdots\times N_d\). The
set of cell indices over this grid is
\begin{equation}
\cellindices=\{1,\ldots, N_1\}\times\cdots\times\{1, \ldots, N_d\},
\end{equation}
(refer to Remark~\ref{rem:20210924090334} for naming conventions of
indices). For \(\tuple{n}, \tuple{p}\in\cellindices\), we define
\(\Phi_{\tuple{n}\tuple{p}}\)
\begin{equation}
\phi_{\tuple{n}\tuple{p}}=2\pi\frac{\bigl(n_1-1\bigr)\bigl(p_1-1\bigr)}{N_1}
+\cdots+2\pi\frac{\bigl(n_d-1\bigr)\bigl(p_d-1\bigr)}{N_d}.
\end{equation}

Note that cell-indices start at 1, hence the \(-1\) correction in the above
formula (see Remark~\ref{rem:20210924090344}).

Let \(x=(x_{\tuple{p}})\) be a finite set of scalar values indexed by the
\(d\)-tuple \(\tuple{p}\in\cellindices\). The discrete Fourier transform is a
discrete set of scalar values \(\dft_{\tuple{n}}(x)\) indexed by the \(d\)-tuple
\(\tuple{n}\in\integers^d\), defined as follows
\begin{equation}
\dft_{\tuple{n}}(x)=\sum_{\tuple{p}\in\cellindices}
\exp\bigl(-\I\phi_{\tuple{n}\tuple{p}}\bigr)x_{\tuple{p}}.
\end{equation}

Note that in the above definition, no restrictions are applied to the
multi-index \(\tuple{n}\). However, it can be verified that the above series of
tensors is in fact \(\tuple{N}\)-periodic:
\(\dft_{\tuple{n}+\tuple{N}}(x)=\dft_{\tuple{n}}(x)\), where
\(\tuple{n}+\tuple{N}=(n_1+N_1, \ldots, n_d+N_d)\). Therefore, the
\(\tuple{n}\)-index is effectively restricted to \(1\leq n_i\leq N_i\) as
well. The most important results concerning the DFT are the \emph{inversion
  formula}
\begin{equation}
x_{\tuple{p}}=\frac1{\lvert\tuple{N}\rvert}\sum_{\tuple{n}\in\cellindices}
\exp\bigl(\I\phi_{\tuple{n}\tuple{p}}\bigr)\dft_{\tuple{n}}(x),
\end{equation}
the \emph{Plancherel theorem}
\begin{equation}
\sum_{\tuple{p}\in\cellindices}\conj(x_{\tuple p})y_{\tuple p}
=\frac1{\lvert\tuple N\rvert}\sum_{\tuple{n}\in\cellindices}
\conj[\dft_{\tuple n}(x)]\dft_{\tuple n}(y),
\end{equation}
and the \emph{circular convolution theorem}
\begin{equation}
\dft_{\tuple{n}}(x\ast y)=\dft_{\tuple{n}}(x)\dft_{\tuple{n}}(y),
\quad\text{where}\quad
(x\ast y)_{\tuple p}=\sum_{\tuple{q}\in\cellindices}
x_{\tuple{q}}y_{\tuple{p}-\tuple{q}+1}.
\end{equation}

The DFT is readily extended to tensor data points. In the absence of ambiguity,
the shorthand \(\hat{x}_{\tuple{n}}\) will be adopted for
\(\dft_{\tuple{n}}(x)\).

To close this section, we observe that the DFT of a series of \emph{real} data
points is a series of \emph{complex} data points. However, these complex values
have the following property
\begin{equation}
\dft_{\tuple{N}-\tuple{n}}(x)=\conj[\dft_{\tuple{n}}(x)],
\end{equation}
which is actually a \emph{necessary and sufficient} condition for the
\(x_{\tuple{p}}\) to be real.
